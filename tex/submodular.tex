\documentclass{classes/report}

\begin{document}
\title{劣モジュラ最適化}
\maketitle
\tableofcontents

\chapter{基礎}

\section{劣モジュラ関数の定義と具体例}

\begin{tcolorbox}
\begin{Def}[劣モジュラ関数 その1] \label{def:submodular-1}
    $n$ 個の要素からなる有限集合 $V=\{1, \dots, n\}$ と、 $V$ を台集合とする集合関数 $f: 2^V \to \mathbb{R}$ を考える。
    $V$ の任意の部分集合 $S, T \subset V$ について次の不等式が成立するとき、 $f$ を劣モジュラ関数と呼ぶ。
    \begin{align}
        f(S) + f(T) \ge f(S \cup T) + f(S \cap T) \label{eq:def-submodular}
    \end{align}
\end{Def}

\begin{Def}[劣モジュラ関数 その2] \label{def:submodular-2}
    $f$ と $V$ は先ほどと同じものとする。
    $S \subset T$ を満たす $V$ の任意の部分集合 $S, T \subset V$ と $T$ に含まれない任意の要素 $i \in V - T$について次の不等式が成立するとき、 $f$ を劣モジュラ関数と呼ぶ。
    \begin{align}
        f(S \cup \{i\}) - f(S) \ge f(T \cup \{i\}) + f(T) \label{eq:diminishing-returns}
    \end{align}
\end{Def}

\ref{eq:diminishing-returns} を限界効用逓減と言う。
\end{tcolorbox}

\begin{prop}[二つの定義の等価性]
    劣モジュラ関数の二つの定義 \ref{def:submodular-1} と \ref{def:submodular-2} は等価である。
\end{prop}

\begin{proof}
    \ref{eq:def-submodular} $\implies$ \ref{eq:diminishing-returns} について。
    $S \subset T, i \in V - T$ とする。
    \begin{align*}
        &f(S \cup \{i\}) + f(T) \ge f((S \cup \{i\}) \cup T) + f((S \cup \{i\}) \cap T) \\
        \implies &f(S \cup \{i\}) + f(T) \ge f(T \cup \{i\}) + f(S) \\
        \implies &f(S \cup \{i\}) - f(S) \ge f(T \cup \{i\}) - f(T) \\
    \end{align*}

    \ref{eq:diminishing-returns} $\implies$ \ref{eq:def-submodular} について。
    $S, T$ を $V$ の任意の部分集合とする。
    $S \subset T$ の場合、\ref{eq:def-submodular} は自明である。
    $S \not \subset T$ の場合を考える。
    $\{i_1, \dots, i_m\} = S - T$ とする。
    集合の増加列を考える:$S_j, T_j (j=0, \dots, m)$ 。
    \begin{align*}
        &S_0 = S \cap T , S_j = S_{j-1} \cup \{i_j\} \\
        &T_0 = T        , T_j = T_{j-1} \cup \{i_j\}
    \end{align*}
    上の式の $j$ は $j=1, \dots, m$ とする。
    $S_m = S, T_m = S \cup T$ が成立する。
    また、 \ref{eq:diminishing-returns} より、$f(S_j) - f(S_{j-1}) \ge f(T_j) - f(T_{j-1})  (j=1, \dots, m)$ が成立する。
    この式を足し合わせることで証明できる。
    \begin{align*}
        &f(S_m) - f(S_0) \ge f(T_m) - f(T_0) \\
        \implies &f(S) - f(S \cap T) \ge f(S \cup T) - f(T) \\
        \implies &f(S) + f(T) \ge f(S \cup T) + f(S \cap T)
    \end{align*}
\end{proof}

\begin{tcolorbox}
    \begin{eg}[カバー関数]
    \end{eg}
    \begin{eg}[グラフのカット関数]
    \end{eg}
    \begin{eg}[凹関数が生成する関数]
    \end{eg}
    \begin{eg}[Matroid rank functions]
    \end{eg}
    \begin{eg}[Entropy functions]
    \end{eg}
\end{tcolorbox}

\section{劣モジュラ多面体と基多面体}

$V=\{1, \dots, n\}, f: 2^V \to \mathbb{R}$ で、 $f$ は正規化された劣モジュラ関数とする
(正規化とは $f(\{\})=0$ が成り立つ関数のこと)。
$f$ を使って、 $\mathbb{R}^n$ に二つの凸多面体 $P(f), B(f)$ を定義できる。

\begin{tcolorbox}
    \begin{Def}[劣モジュラ多面体、基多面体]
        $x=(x_1, \dots, x_n)$ を $n$ 次元の変数ベクトルとし、 $V$ の各部分集合を $S$ とする。
        $x(S) = \sum_{i \in S} x_i$ として、
        劣モジュラ多面体 (submodular polyhedron) $P(f)$ を次のように定義する。
        \begin{align}
            P(f) = \{ x \in \mathbb{R}^n | x(S) \le f(S), S \subset V \}
        \end{align}
        また、基多面体 (base polytope) $B(f)$ は次のように定義する。
        \begin{align}
            P(f) = \{ x \in P(f) | x(V) = f(V) \}
        \end{align}
    \end{Def}
\end{tcolorbox}

\begin{prop}
    $B(f)$ は有界
\end{prop}

\begin{proof}
    \begin{align*}
        &x_i \le f(\{i\}) \\
        &x(V-\{i\}) \le f(V-\{i\}) \iff x(V) - x_i \le f(V-\{i\}) \\
        &x(V) = f(V) \\
        \implies &f(V) - f(V-\{i\}) \le x_i \le f(\{i\})
    \end{align*}
\end{proof}

\begin{prop}
    $B(f)$ の端点と $P(f)$ の端点は一致する。
\end{prop}

集合 $V$ を並び替えた線形順序 (linear ordering) $L=(i_1, \dots, i_n)$ を考える。 $n!$ 通りある。
この時、 $L$ に対応する基多面体 $B(f)$ の端点 $x^L$ を定めることができる。

\section{ロヴァース拡張}

\chapter{劣モジュラ最適化}

\chapter{実装}

\begin{itemize}
    \item \url{https://github.com/OptMist-Tokyo/SubmodularFunctionMinimization}
    \item \url{https://github.com/joschout/SubmodularMaximization}
\end{itemize}

\chapter{参考文献}

\begin{itemize}
    \item \url{https://proceedings.mlr.press/v119/breuer20a.html}
    \item \url{https://fujiik.github.io/}
    \item \url{https://viterbi-web.usc.edu/~shanghua/teaching/Fall2021-670/krause12survey.pdf}
    \item \url{https://www.kurims.kyoto-u.ac.jp/preprint/file/RIMS1634.pdf}
    \item \url{http://papers.neurips.cc/paper/6652-continuous-dr-submodular-maximization-structure-and-algorithms.pdf}
\end{itemize}

\end{document}
